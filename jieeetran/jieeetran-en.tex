\documentclass[11pt, a4paper, dvipdfmx]{article}
\usepackage{lmodern}
\usepackage{bxtexlogo}

\title{\texttt{jIEEEtran.bst}:\\Unofficial IEEE \BibTeX\ style for Japanese\\(ver. 0.17)}
\author{Haruki EJIRI}

\begin{document}

\maketitle

\begin{abstract}
The jIEEEtran \BibTeX\ style is an unofficial IEEE style citation format for Japanese users.
It is a customization of Michael Shell's IEEEtran.bst, and provides natural Japanese name handling.
A Python script named mixej.py enables one content with both English and Japanese format to be handled properly.
\end{abstract}


\section{Combine English and Japanese entries}

Japanese users are sometimes required to write both English and Japanese citation formats as one bibliography entry.
A python script named \texttt{mixej.py} included in the \texttt{jieeetran} package enables combining English and Japanese entries.
To combine two entries of enkey and jpkey, users have to cite as \texttt{\string\cite\string{enkye/ej/jpkey\string}}, where enkey and jpkey are English and Japanese citation keys, respectively.
Then compile by the following seven steps:
\begin{center}
    \texttt{uplatex → python mixej.py → upbibtex → python mixej.py\\→ uplatex → uplatex → dvipdfmx}
\end{center}
instead of the typical five steps:
\begin{center}
    \texttt{uplatex → upbibtex → uplatex → uplatex → dvipdfmx}.
\end{center}
Note that the uplatex command is Unicode and Japanese compatible latex command.

The Japanese version of this document, \texttt{jieeetran.pdf}, shows English and Japanese combined citation examples.

\end{document}
