\documentclass[11pt, a4paper, dvipdfmx, uplatex]{jsarticle}

\title{\texttt{IEEJtran.bst}:\\電気学会非公式\BibTeX スタイル\\(ver. 0.17)}
\author{江尻 開}

\begin{document}

\maketitle

\begin{abstract}
IEEJtranは電気学会形式の非公式\BibTeX スタイルです。
Michael Shell氏のIEEEtran.bstを基に開発されています。
一つのエントリに英語と日本語を併記可能とするPythonスクリプトmixej.pyも提供します。
\end{abstract}


\section{はじめに}

\texttt{IEEJtran.bst}は電気系の論文を執筆する日本人ユーザ向けに開発された電気学会形式の非公式\BibTeX スタイル(\texttt{.bst})ファイルです。
電気学会論文誌の下書きや卒業論文,修士論文など,\LaTeX を使用した文献管理の一助になることを目的としています。
Michael Shell氏による\texttt{IEEEtran.bst}\cite{IEEEtran}をもとにYoshiRi氏が日本語の自然な表記を実装し,現在は著者がリポジトリを譲渡していただきGitHub上\cite{jIEEEtran}で継続して開発しています。
同レポジトリではIEEEの引用スタイルを保ったまま日本語の取り扱いを自然にした\BibTeX スタイルである\texttt{jIEEEtran.bst}も取り扱っています。
オプションの追記方法や日本語の判別方法については,武田氏らによる\texttt{jecon.bst}\cite{jeconbst}を参考にしています。
間違いがあった場合でも大きな被害が出るレポジトリではありませんので,ご要望や問題があればお気軽にIssueやPull Requestを飛ばしてください。
なお,電気学会公認ではありませんし,全ての環境で動作を保証するものではありません。
不具合が生じても責任を負うことはできませんのでご容赦ください。
なお,\TeX Live 2019以降の使用を想定しています。
\texttt{w32tex}や古い\TeX Liveでの動作は確認していません。
\TeX エンジンとしてplatex/pbibtexもしくはuplatex/upbibtexを推奨し,本マニュアルはuplatexでコンパイルしています。
\TeX エンジンが出力する中間ファイルの改変にPythonを使用します。
Pythonは\texttt{3.7}以降での動作を確認しています。


\section{使用方法}

\texttt{IEEJtran.bst}が\TeX Liveのパッケージマネージャ等でインストールされ,PATHの通った場所に配置されている場合,プリアンブルに\texttt{\string\bibliographystyle\string{IEEJtran\string}}と記載すれば設定は完了です。
後述のカスタマイズ等の目的で\texttt{IEEJtran.bst}を個別にダウンロードして配置する場合,メインの\TeX ファイルと同じディレクトリに格納し,同様にプリアンブルに\texttt{\string\bibliographystyle\string{IEEJtran\string}}と記載すれば設定は完了です。

文献情報が記載される\texttt{.bib}ファイル上では,日本人著者は
\texttt{author = \string{\string{\textgt{姓姓 名名}\string} and \string{\textgt{姓姓 名名}\string}\string}}と,\texttt{\string{\string}}で挟むか,\texttt{author = \string{姓姓, 名名 and 姓姓, 名名\string}}と,\texttt{\string{\string}}で挟まず半角カンマを挿入して記入してください。


\section{英語と日本語の併記方法}

電気学会の引用スタイルで求められているように,一つの文献エントリに英語と日本語の情報を併記する場合,\texttt{.tex}ファイルには次のように記載してください。
\begin{center}
  \texttt{\string\cite\string{enArticle1/ej/jpArticle1\string}}
\end{center}
加えて,外部pythonファイルの\texttt{mixej.py}を用いて\texttt{.aux}ファイルと\texttt{.bbl}ファイルを改変する必要があります。
具体的には,\BibTeX コマンドの実行前に\texttt{\string\cite\string{enArticle1/ej/jpArticle1\string}}を\texttt{enArticle1}と\texttt{jpArticle1}の2つのエントリに分け,\BibTeX コマンドが文献情報を書した後に再度結合させる操作を行います。


\end{document}
